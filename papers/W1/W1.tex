% Created 2025-02-13 Thu 13:56
% Intended LaTeX compiler: pdflatex
\documentclass[11pt]{article}
\usepackage[utf8]{inputenc}
\usepackage[T1]{fontenc}
\usepackage{amsmath}
\usepackage{amssymb}
\usepackage{capt-of}
\usepackage{hyperref}
\usepackage{physics}
\usepackage[per-mode=fraction]{siunitx}
\usepackage[margin=1.25in]{geometry}

%% ox-latex features:
%   !announce-start, !guess-pollyglossia, !guess-babel, !guess-inputenc, maths,
%   !announce-end.

\usepackage{amsmath}
\usepackage{amssymb}

%% end ox-latex features


\author{Jesse Young Lin}
\date{\today}
\title{Markov Chain Monte Carlo Notes}
\hypersetup{
 pdfauthor={Jesse Young Lin},
 pdftitle={Markov Chain Monte Carlo Notes},
 pdfkeywords={},
 pdfsubject={},
 pdfcreator={Emacs 29.4 (Org mode 9.8-pre)},
 pdflang={English}}
\begin{document}

\maketitle
\tableofcontents

\section{Preliminaries on Probability}
\label{sec:org22a565b}
Probability is extremely unintuitive for humans. Consequently,
there is a variety of vocabulary and notation which is initially confusing but
essential to understand. We will illustrate all the preliminaries with the
example of a fair coin flip (which is formally known as a Bernoulli distributed
random variable).
\subsection{Probability distributions}
\label{sec:org6627e45}
A \textbf{random variable} is typically denoted with a capital letter, and its
\textbf{realization} is often denoted with the lowercase letter. To model a series of
coin flips, we consider a sequence of \(N\) random variables
\(\{X_{i}\}_{i=0}^{N}\). Say we flip the \(i=0\) coin and see heads. Then, we
say that \(x_{0} = H\).

The \textbf{probability distribution} or \textbf{probability density function} of \(X_{0}\) is
given by the Bernoulli distribution, which is a function \(\rho\)
\begin{align*}
\rho(x) = \begin{cases}
1/2\qc{x = H} \\
1/2\qc{x = T}
\end{cases}.
\end{align*}
This communicates the intuitive fact that if we flip a fair coin a large number
of times, we expect it to land on heads half the time and tails the other half.
We note that \(\rho(x)\) is just a function over the set \(\{H,T\}\), its
probabilistic meaning is given by the notation
\begin{align*}
\mathbb{P}(X_{0} = z) = \rho(z).
\end{align*}
We use the \(\mathbb{P}\) notation to describe the probability of particular
realizations. For example, if \(z = H\), then the above equation says that the
probability the random variable \(X_{0}\) has value \(H\) is given by the value
of the function \(\rho(z)\) at \(z = H\), which is \(1/2\).
\subsection{Conditional probability and (in)dependence of random variables}
\label{sec:org275793a}
In our sequence of coin flips, we have considered the first coin \(X_{0}\). What
about the \(X_{1}\) variable? Now, we need to introduce the assumption that
\(X_{1}\) is \textbf{independent} of \(X_{0}\). Intuitively, this means that the
realization (i.e., the result of the coin flip) \(x_{1}\) does not depend on
information from the realization \(x_{0}\). Formally, we write
\begin{align}
\label{eq:independence}
\mathbb{P}(X_{1} | X_{0}) = \mathbb{P}(X_{1}).
\end{align}
The left-hand side of this equation is the \textbf{conditional probability}. It is read
as the conditional probability of \(X_{1}\) conditioned on the realization of
\(X_{0}\). In words, what is the probability which describes \(X_{1}\) given
that we know the realization of \(X_{0}\)? As we are modelling a sequence of
coin flips, we know that the probability distribution of the \(i=1\) should have
nothing to do with whether the coin flip at \(i = 0\) was heads or tails, which
is what \eqref{eq:independence} tells us. Finally, we have been assuming that all the
\(X_{i}\) are identically distributed, i.e., they are all Bernoulli distributed,
which tells us that
\begin{align*}
\mathbb{P}(X_{i} = z) = \rho(z)
\end{align*}
for any \(i\). Therefore we have our final model of a sequence of coin flips,
which is given by the sequence \(\{X_{i}\}\) of independently and identically
distributed random variables, all distributed according to the Bernoulli
distribution.\footnote{We note also that independence is often defined via the following equation on
the \textbf{joint probability}
\begin{align*}
\mathbb{P}(X_{1} = x\text{ and }X_{0} = y) = \mathbb{P}(X_{1} = x)\mathbb{P}(X_{0} = y).
\end{align*}
This is equivalent to \eqref{eq:independence} if we use \textbf{Bayes' theorem}, which
is the following statement which always holds
\begin{align*}
\mathbb{P}(X_{1} = x \text{ and } X_{0} = y) = \mathbb{P}(X_{1} = x \mid X_{0} = y)\mathbb{P}(X_{0} = y).
\end{align*}
If we denote using \(A\) and \(B\) the realizations
\begin{align*}
A &= \{X_{1} = x\} \\
B &= \{X_{0} = y\}
\end{align*}
and use the set theoretic notation for ``and'', we get the common expression
\begin{align*}
\mathbb{P}(A \cap B) = \mathbb{P}(A \mid B)\mathbb{P}(B).
\end{align*}
This equation says that the probability of events \(A\) and \(B\) occurring
simultaneously is equal to the probability that \(B\) occurs multiplied by the
probability that \(A\) occurs given that we know event \(B\) occurs (the
conditional probability).}

A simple example of dependent random variables is the following. Imagine an urn
filled with one red ball and one black ball. Let \(Y_{i}\) be the random variable
corresponding to the \(i\)-th draw from the urn. Now,
\begin{align*}
\mathbb{P}(Y_{1} = \text{red} \mid Y_{0} = \text{red}) &= 0 \\
\mathbb{P}(Y_{1} = \text{black} \mid Y_{0} = \text{red}) &= 1,
\end{align*}
in other words if you first draw red then you know the next draw must be black.
However
\begin{align*}
\mathbb{P}(Y_{1} = \text{red}) = 1/2\\
\mathbb{P}(Y_{1} = \text{black}) = 1/2.
\end{align*}
which means if you just consider drawing from the urn twice, the second draw has
a uniform probability of being either the red or the black one. This
violates the equation \eqref{eq:independence}.
\subsection{Expectation value}
\label{sec:orgcc4878a}
The final concept is the most intuitive one. In the coin flip example we denoted
the heads and tails by symbols \(\{H, T\}\). Let's imagine a game where
everytime you flip heads you gain 1 dollar and everytime you flip tails you lose
1 dollar. This is equivalent to assigning numbers \(H \to 1\) and \(T \to -1\).
Let the sequence \(\{Z_{i}\}\) of random variables correspond to the payoff at
each step \(i\) of this game. The average payoff at any step \(i\) is evidently
\(0\). Formally this is denoted with the \textbf{expectation value} defined as follows
\begin{align*}
\mathbb{E}(Z_{i}) &= \sum_{z \in \{\pm 1\}} z \mathbb{P}(Z_{i} = z)
\end{align*}
which is easy to compute:
\begin{align*}
\sum_{z \in \{\pm 1\}} z \mathbb{P}(Z_{i} = z)
                 &= \sum_{z \in \{\pm 1\}} z \rho(z) \\
                 &= (1)(1/2) + (-1)(1/2) \\
                 &= 0.
\end{align*}
\section{Markov Chains}
\label{sec:orgf129e5f}
\subsection{Definition}
\label{sec:org71ab151}
A Markov chain is a sequence of random variables which is \textbf{memoryless}. In other
words, for a sequence \(\{X_{i}\}\)
\begin{align}
\label{eq:markov}
\mathbb{P}(X_{j} \mid X_{j-1}, X_{j-2}, \ldots, X_{0}) = \mathbb{P}(X_{j} \mid X_{j-1}).
\end{align}
Intuitively, the value of random variable \(X\) at time \(j\) depends only on
its value at the previous time \(j-1\) and it has no memory of the history
before that. Equation \eqref{eq:markov} is called the \textbf{Markov property}. Often
we define the \textbf{Markov transition matrix}
\begin{align*}
W(x, y) = \mathbb{P}(X_{j} = x \mid X_{j-1} = y).
\end{align*}

The essential feature of Markovian systems is the ability to predict
the future given an initial condition by repeated application of \(W\), i.e.,
\begin{align}
\label{eq:propagator}
\mathbb{P}(X_{n} = x_{n} \mid X_{0} = x_{0}) &= \sum_{\{x_{n-1}, \ldots, x_{1}\}} W(x_{n}, x_{n-1}) \cdots W(x_{1}, x_{0})\mathbb{P}(X_{0} = x_{0}).
\end{align}
A derivation is given in \footnote{Bayes' theorem allows the following decomposition of the joint probabilities
\begin{align*}
\mathbb{P}(X_{n}, \ldots, X_{0}) &= \sum_{\{x_{n-1}, \ldots, x_{0}\}}\mathbb{P}(X_{n} \mid X_{n-1}= x_{n-1}, \ldots, X_{0} = x_{0}) \mathbb{P}(X_{n-1} = x_{n-1}, \ldots, X_{0} = x_{0})
\end{align*}
where the sum is taken over all possible values of the \(\{x_{n-1}, \ldots,
x_{0}\}\), e.g., if each \(X_{i}\) models a coin flip then
\begin{align*}
\sum_{\{x_{n-1}, \ldots, x_{0}\}} = \sum_{x_{n-1} \in \{H, T\}} \cdots \sum_{x_{0} \in \{H,T\}}.
\end{align*}
Then,
\begin{align*}
\mathbb{P}(X_{n} = x_{n} \mid X_{0} = x_{0}) &= \sum \mathbb{P}(X_{n} = x_{n} \mid X_{n-1} = x_{n-1}, \ldots, X_{0} = x_{0}) \mathbb{P}(X_{n-1} = x_{n-1}, \ldots, X_{0} - x_{0}) \\
&= \sum \mathbb{P}(X_{n} = x_{n} \mid X_{n-1} = x_{n-1}) \mathbb{P}(X_{n-1} = x_{n-1}, \ldots, X_{0} = x_{0}) \\
&= \sum W(x_{n}, x_{n-1}) \mathbb{P}(X_{n-1} = x_{n-1}, \ldots, X_{0} = x_{0}) \\
&= \sum W(x_{n}, x_{n-1}) \cdots W(x_{1}, x_{0})\mathbb{P}(X_{0} = x_{0}) \\
&= \sum_{x_{0}} W^{n}(x_{n},x_{0}) \mathbb{P}(X_{0} = x_{0})
\end{align*}}. The above when applied to quantum mechanics is
actually the celebrated Feynman path integral.
\subsection{Equilibrium}
\label{sec:org8f32af4}
If the values \(x_{i}\) in \eqref{eq:propagator} are taken to assume only
finitely many values, then we can write the equivalent matrix-vector equation
\begin{align*}
P_{t} = W^{t} P_{0}
\end{align*}
where \(P_{t}\) is the vector of probabilities at time \(t\), and \(W\) is a
matrix. The superscript \(t\) represents the repeated multplication of the \(W\).

The \textbf{equilibrium} or \textbf{invariant distribution} of a Markov chain is the
probability distribution \(P\) which satisfies
\begin{align}
\label{eq:eigenvalue}
P = WP.
\end{align}
In linear algebra, this condition \eqref{eq:eigenvalue} is known as an \textbf{eigenvalue equation}. One
approach to solving the above is to consider the components
\begin{align*}
P_{i} = \sum_{j} W_{ij} P_{j}
\end{align*}
then using the fact that \(\sum_{j} W_{ji} = 1\) (i.e., the transition matrix
must conserve probability), this is equivalent to
\begin{align*}
\sum_{j} W_{ji} P_{i} = \sum_{j} W_{ij}P_{j}
\end{align*}
and one way to solve this is with a vector \(P\) that satisfies, for each component
\begin{align}
\label{eq:detailedbalance}
W_{ji}P_{i} = W_{ij}P_{j}.
\end{align}
The condition \eqref{eq:detailedbalance} is called \textbf{detailed balance}. It
indicates that, at all times, the rate of transitions between states \(i \to j\)
is exactly balanced by the rate of transitions \(j \to i\). Intuitively, then,
the probability \(P_{k}\) of being in any state \(k\) must be constant in time.
\section{Markov Chain Monte Carlo}
\label{sec:org093bac0}
The essential idea is now immediate to state: to sample from a complex
probability distribution \(P\), it is often easier to design the transition
matrix \(W\) of a Markov chain such that \(P\) is its invariant distribution.
Then, independent simulations of the Markov chain can be done on a computer, and
given sufficient time one expects that the simulated data obeys \(P\).

Designing a Markov chain is often conceptually simple: for example, to sample
from a chemical system in equilibrium a transition matrix which satisfies
detailed balance is given directly by the kinetic rates and stochiometry of the
reactants. Markov chains for most systems also benefit from an \textbf{exponential}
convergence rate to equilibrium, which means the algorithm is usually quite
efficient. \footnote{A notable exception occurs with systems which are at a critical point.
This is an extremely rich subject, especially in study of the Ising model. It's
out of scope of our project but I encourage looking it up!}

The algorithm we use is the \textbf{Metropolis-Hastings} algorithm. It is essentially a
specification of the transition matrix \(W\). There are other choices for \(W\),
such as the \textbf{Gibbs sampler}, but the basic idea is the same.
\end{document}
