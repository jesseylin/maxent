% Created 2025-02-14 Fri 11:29
% Intended LaTeX compiler: pdflatex
\documentclass[11pt]{article}
\usepackage[utf8]{inputenc}
\usepackage[T1]{fontenc}
\usepackage{amsmath}
\usepackage{amssymb}
\usepackage{capt-of}
\usepackage{hyperref}
\usepackage{physics}
\usepackage[per-mode=fraction]{siunitx}
\usepackage[margin=1.25in]{geometry}
\DeclareMathOperator*{\argmax}{arg\,max}

%% ox-latex features:
%   !announce-start, !guess-pollyglossia, !guess-babel, !guess-inputenc, maths,
%   !announce-end.

\usepackage{amsmath}
\usepackage{amssymb}

%% end ox-latex features


\author{Jesse Young Lin}
\date{\today}
\title{Inverse Statistical Mechanics Notes}
\hypersetup{
 pdfauthor={Jesse Young Lin},
 pdftitle={Inverse Statistical Mechanics Notes},
 pdfkeywords={},
 pdfsubject={},
 pdfcreator={Emacs 29.4 (Org mode 9.8-pre)},
 pdflang={English}}
\begin{document}

\maketitle
\tableofcontents

\section{Bayesian statistics}
\label{sec:orgc31e2eb}
\subsection{The question of Bayesian inference and maximum likelihood estimation}
\label{sec:orgbc8ce79}
The core question is, given high-dimensional data, i.e., \(\{x_{i}\}\) with each
\(x_{i} \in \mathbb{R}^{d}\) and \(d\) very large, how does one fit a
probability distribution \(\rho(\{x_{i}\})\) over the data set which ``respects''
its statistics?

In general, this is the question answered by \textbf{Bayesian inference}. A common
paradigm is \textbf{maximum likelihood estimation}. Here, one considers a probablity
distribution \(\rho\) which is parameterized by variables \(\{\theta_{j}\}\)
with allowable values in the sets \(\{\Theta_{j}\}\).
Then, the \textbf{likelihood function} \(\rho( \{x_{i}\} \mid \{\theta_{j}\})\) is
defined over the data. Its interpretation is as a conditional probability: what
is the probability of observing my dataset \(\{x_{i}\}\) given specific values
of the unknown fit parameters \(\{\theta_{j}\}\)? Maximum likelihood estimation
is a way of estimating the \(\{\theta_{j}\}\) by assuming that they take on the
values which maximize the likelihood, i.e.,
\begin{align*}
\theta_{j} = \argmax_{\theta_{j} \in \Theta_{j}} \rho(\{x_{i}\} \mid \{\theta_{j}\}).
\end{align*}
\subsection{Maximum entropy inference}
\label{sec:org282131b}
We are using a specific technique for doing maximum likelihood estimation called
\textbf{maximum entropy inference}. Essentially, we posit a maximum entropy assumption
when defining our model distribution \(\rho\), which otherwise must be
arbitrarily specified.

For concreteness, consider the Ising model with random variables \(S_{i}\). If
we denote a lattice configuration with size \(N \times N\) as a vector\footnote{Physically we imagine the \(N \times N\) lattice like a matrix, but
mathematically for convenience we denote it as a vector of length \(N^{2}\). In
computer science this is sometimes called ``flattening'' a matrix, and can be
convenient when programming high-performance simulations because computer memory is
linear and therefore matrices are all stored in flattened form. There are many
ways to flatten a matrix, for example: for linear index \(k\), doing division
gives integers \(q,r\) such that
\begin{align*}
k = qN + r.
\end{align*}
with \(0 \leq r < q\).
Then the Cartesian index (i.e., \(x\)- and \(y\)-coordinates on the lattice) is \((q,r)\).}
\begin{align*}
\Lambda = (s_{1}, s_{2}, \ldots, s_{N \times N})
\end{align*}
i.e., we specify the values of each of the random variables as \(s_{i}\), then a
dataset is a set of lattice configurations \(\{\Lambda^{j}\}_{j=1}^{M}\).
Therefore, we use subscript index to denote the lattice position and superscript
index to denote which sample in the dataset we refer to.

Given such a dataset \(\{\Lambda^{j}\}_{j=1}^{M}\), we want to fit a probability model \(\rho\)
which reproduces the first and second moments of the dataset.\footnote{Note that due to the symmetries of the Ising model, namely its spatial
homogeneity (no spatial position is special) and isotropy (no spatial direction
is special), many of these moments are actually the same. For example,
\(\ev{S_{i}}\) is actually independent of the lattice position \(i\): this
is the physical statement that the net magnetization is uniform across the
lattice.}
In other words, we want
\begin{align}
\label{eq:moments}
\begin{split}
\ev{S_{i}}_{\rho} &= \frac{1}{M}\sum_{k=1}^{M} s^{k}_{i} \\
\ev{S_{i} S_{j}}_{\rho} &= \frac{1}{M}\sum_{k=1}^{M} s_{i}^{k}s^{k}_{j},
\end{split}
\end{align}
where \(\ev{\cdot}_{\rho}\) denotes the expectation value with respect to the
distribution \(\rho\).

It turns out that matching the moments via \eqref{eq:moments} does not uniquely
specify the probability distribution, but if we also add the constraint that the
entropy is maximized, the unique probability distribution that satisfies
\eqref{eq:moments} with maximal entropy is
\begin{align}
\label{eq:gibbs}
\rho(\Lambda) = Z^{-1}e^{-\sum_{ij} J_{ij}s_{i}s_{j} -\sum_{k} h_{k}s_{k}}.
\end{align}
Defining the \textbf{partition function} as\footnote{I prefer writing the summation as over all possible configurations of the
lattice \(\Lambda\). For the Ising model, which is a 2D lattice of binary spins,
this is a summation over a Cartesian product \(\Lambda \in \{\pm 1\}^{N^{2}}\).}
\begin{align*}
Z = \sum_{\text{all possible } \Lambda} \rho(\Lambda),
\end{align*}
we note
\begin{align*}
\dv{\log Z}{J_{ij}} &= \ev{S_{i}S_{j}}_{\rho} \\
\dv{\log Z}{h_{k}} &= \ev{S_{k}}_{\rho}.
\end{align*}

Therefore the condition \eqref{eq:moments} is equivalent to
\begin{align*}
\dv{\log Z}{J_{ij}} &= \frac{1}{M}\sum_{k=1}^{M} s^{k}_{i} \\
\dv{\log Z}{h_{k}} &= \frac{1}{M}\sum_{k=1}^{M} s_{i}^{k}s^{k}_{j}.
\end{align*}
\section{Gradient descent and machine learning}
\label{sec:org1980ee7}
Essentially all the success of machine learning is based on the fact that the
gradient descent algorithm is very efficient to compute. Gradient descent
amounts to defining a cost function \(\mathcal L(\theta)\) over a parameter
\(\theta\), and then noting that if
\begin{align}
\label{eq:gradientdescent}
\theta \leftarrow \theta - \alpha \pdv{\mathcal L}{\theta}.
\end{align}
where \(\alpha > 0\) is the dimensionless \textbf{step size}, then \(\theta\) will
converge to a value where \(\pdv{\mathcal{L}}{\theta} = 0\). This corresponds to
a local extremum, and the choice of minus sign in the algorithm
\eqref{eq:gradientdescent} indicates that this should be a local minimum of
\(\mathcal L\). If \(\mathcal{L}\) fulfills other conditions (for example, if it were
a globally convex function of \(\theta\)), then this will also be a global minimum, but
in general it does not fulfill these conditions and we are satisfied if we can
even find local minima.
\subsection{Regression on moments is maximum likelihood estimation}
\label{sec:orgcb847f5}
Let's write the distribution \eqref{eq:gibbs} as a likelihood function,
\begin{align*}
\rho(\{s_{i}\} \mid \{J_{ij}\}, \{h_{k}\}) = Z^{-1}(\{J_{ij}\}, \{h_{k}\})
e^{-\sum_{ij} J_{ij}s_{i}s_{j} -\sum_{k} h_{k}s_{k}}.
\end{align*}
This means that we fix some lattice configuration given by \(\{s_{i}\}\), and
then this function \(\rho\) gives us its probability given some chosen values
for the parameters \(\{J_{ij}, h_{k}\}\).

Now, one can calculate
\begin{align*}
-\pdv{\log \rho(\{s_{i}\} \mid \{J_{ij}\}, \{h_{k}\})}{J_{ij}} &= \ev{S_{i}S_{j}}_{\rho} - s_{i}s_{j} \\
-\pdv{\log \rho(\{s_{i}\} \mid \{J_{ij}\}, \{h_{k}\})}{h_{k}} &= \ev{S_{k}}_{\rho} - s_{k}
\end{align*}
which we saw on the board last time. Therefore, we can implement the following algorithm
\begin{align}
\label{eq:mledyn}
\begin{split}
J_{ij} \leftarrow J_{ij} - \alpha(\ev{S_{i}S_{j}}_{\rho} - s_{i}s_{j}) \\
h_{k} \leftarrow h_{k} - \alpha(\ev{S_{k}}_{\rho} - s_{k})
\end{split}
\end{align}
which is equivalent to gradient ascent on the function \(\log \rho\), i.e.,
maximum likelihood estimation.\footnote{We are now talking about an algorithm which updates the parameters
\(\{J_{ij}, h_{k}\}\) as part of a statistical inference procedure. We briefly
discussed that there is actually there is a physical interpretation: one can
consider a lattice spin model where the spin variables \(s_{i}\) thermally
equilibriate on some timescale much shorter than when the interaction or
``disorder'' variables \(J_{ij}\) equilibriate. This can be used to model the
formation of glass, and was the subject of the 2021 Nobel Prize in Physics won
by Giorgio Parisi.}
\subsection{Assignment}
\label{sec:org9db5d3b}
You will create a dataset \(\{\Lambda^{k}\}\) of binary images (it can even be a
single image). Then, code an algorithm which implements \eqref{eq:mledyn} for
the Ising model, starting from randomly initialized parameters \(\{J_{ij},
h_{k}\}\).

Recall, for the Ising model that \(J_{ij}\) is zero unless
sites \(i\) and \(j\) are adjacent on the lattice, and if so then \(J_{ij}\)
equals a single constant \(J\). The code we are using does not have the term
that depends on \(h_{k}\). You should add it into the code, with the same
assumption that \(h_{k}\) is uniform across the lattice with
\(h_{k} = \frac{h}{N^{2}}\). (Note: if you work out the sums in
\eqref{eq:gibbs} you will see where the factors of \(N\) come from. This is not
that important because you are fitting the values of \(h\) and \(J\) anyway.)

Hints:
\begin{itemize}
\item The easiest dataset you could consider is just to pick a value of \(J\) and use the existing code to generate images for your \(\Lambda^{k}\). Then when you do maximum likelihood estimation, you should converge on the original value of \(J\) you chose. This is bona fide "inverse statistical mechanics".
\item Alternatively, you can find your own images from the internet. You may need an image processing library like \url{https://pypi.org/project/pillow/}. Then, one needs to crop and subsample your images down to the resolution \(N \times N\) of the lattice, and also binarize them so they consist of pixels only of brightness value \(0\) or \(1\) (corresponding to the up and down spins of the Ising lattice).
\item If you are using a dataset with more than a single image, it may be worth implementing the algorithm as follows:
\begin{align*}
J_{ij} \leftarrow J_{ij} - \alpha \qty(\ev{S_{i}S_{j}}_{\rho} - \frac{1}{M} \sum_{k} s^{k}_{i}s^{k}_{j}) \\
h_{k} \leftarrow h_{k} - \alpha \qty(\ev{S_{k}}_{\rho} - \frac{1}{M} \sum_{\ell} s^{\ell}_{k}).
\end{align*}

This produces statistically equivalent results as \eqref{eq:mledyn} assuming all your data are "independent". As we discussed, it is not always straightforward to determine when real-world data should be considered independent from each other. Note that playing around with the value of the step size \(\alpha\) may be needed for best results.
\end{itemize}
\end{document}
